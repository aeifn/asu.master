% http://coursera.org/course/latex
% Исходная версия Шаблона --- 
% https://www.writelatex.com/coursera/latex/1.1


\documentclass[a4paper,12pt]{article}

\usepackage{cmap}					% поиск в PDF
\usepackage[T2A]{fontenc}			% кодировка
\usepackage[utf8]{inputenc}			% кодировка исходного текста
\usepackage[english,russian]{babel}	% локализация и переносы

\author{Егор Кузьмичев, Адыгейский государственный университет}
\title{1.1 Наш первый документ}
\date{\today}

\begin{document} % Конец преамбулы, начало текста.

\maketitle

Задачей этого эссе является познакомиться с алгоритмической теорией игр и ответить на вопросы:
\begin{enumerate}
\item Что такое алгоритмическая теория игр?
\item Какие основные труды существют по АГТ?
\item Какое место АГТ занимает в теории игр и история ее создания
\item Какие применения она имеет?
\item Какие основные теоремы АГТ?
\end{enumerate}

Основные книги для эссе --- \cite{agt}


\section{Что такое алгоритмическая теория игр?}

Рассматриваются игры со множеством игроков (тысячи и миллионы). 

\section{Основные труды по теории игр}
Теория игр и алгоритмы разрабатывались Джоном фон Нейманом \cite{neumann}.

\section{Глоссарий}

Дилемма заключенного (пример с ISP)

The Tragedy of the Commons (<<Трагедия общих ресурсов>>)
(Например, перегруженный общий прокси-сервер)

Pollution game --- случай дилеммы заключенного для многих игроков

Coordination Games
--- Battle of sexes

Равновесие Нэша

Игры с полной информацией

Игры с неполной информацией (Байесовские игры).


\section{Применение}
\subsection{ISP routing game}
Интересно применение в интернете. Распределение трафика в подсетях своих и чужих.

\medskip

\bibliographystyle{alpha}%Used BibTeX style is unsrt
\bibliography{sample}

\end{document} % Конец текста.

