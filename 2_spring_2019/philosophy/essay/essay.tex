% http://coursera.org/course/latex
\documentclass[a4paper,14pt]{article}
%%% Работа с русским языком
\usepackage{cmap}					% поиск в PDF
\usepackage{mathtext} 				% русские буквы в фомулах
\usepackage[T2A]{fontenc}			% кодировка
\usepackage[utf8]{inputenc}			% кодировка исходного текста
\usepackage[english,russian]{babel}	% локализация и переносы

\author{Егор Кузьмичёв}

%%% Дополнительная работа с математикой
\usepackage{amsmath,amsfonts,amssymb,amsthm,mathtools} % AMS
\usepackage{icomma} % "Умная" запятая: $0,2$ --- число, $0, 2$ --- перечисление

%% Номера формул
%\mathtoolsset{showonlyrefs=true} % Показывать номера только у тех формул, на которые есть \eqref{} в тексте.
%\usepackage{leqno} % Немуреация формул слева


\usepackage{hyperref}
\usepackage[usenames,dvipsnames,svgnames,table,rgb]{xcolor}
\hypersetup{				% Гиперссылки
    unicode=true,           % русские буквы в раздела PDF
    pdftitle={Заголовок},   % Заголовок
    pdfauthor={Автор},      % Автор
    pdfsubject={Тема},      % Тема
    pdfcreator={Создатель}, % Создатель
    pdfproducer={Производитель}, % Производитель
    pdfkeywords={keyword1} {key2} {key3}, % Ключевые слова
    colorlinks=true,       	% false: ссылки в рамках; true: цветные ссылки
    linkcolor=red,          % внутренние ссылки
    citecolor=black,        % на библиографию
    filecolor=magenta,      % на файлы
    urlcolor=cyan           % на URL
}

\usepackage{csquotes} % Инструменты для ссылок

% https://www.overleaf.com/learn/latex/Biblatex_bibliography_styles
\usepackage[
	backend=biber,
	bibencoding=utf8,
	sorting=ynt,
	maxcitenames=2,
	style=numeric
]{biblatex}


% Требоване: поля 2,5 см.
\usepackage[left=2.5cm,right=2.5cm,
    top=2.5cm,bottom=2.5cm,bindingoffset=0cm]{geometry}
\usepackage{datetime}
\newdateformat{yearf}{\THEYEAR}

% Требование: интервал 1.5
% https://proft.me/2013/06/9/latex-ukazanie-mezhstrochnogo-intervala/ 
\usepackage{setspace}
\onehalfspacing

\addto\captionsrussian{% Replace "english" with the language you use
  \renewcommand{\contentsname}%
    {Оглавление}%
}

\bibliography{essay}


\usepackage{multicol} % Несколько колонок
\usepackage{graphicx}
\graphicspath{ {./images/} }


\author{Кузьмичёв Е. А.}
\title{Взгляды математиков на методологические проблемы науки}
\def \subtitle{Г. Кантор, Д. Гильберт, А. Пуанкаре, Г. Вейль, Н. Н. Лузин, А. Н. Колмогоров, В. И. Арнольд, С. П. Новиков}

\date{\today}


\begin{document}

% https://tex.stackexchange.com/questions/10130/use-the-values-of-title-author-and-date-on-a-custom-title-page
\makeatletter
\begin{titlepage}
\newpage

\begin{center}
Министерство науки и высшего образования РФ \\
Адыгейский государственный университет
\end{center}

\vspace{8em}

\begin{center}
\Large Кафедра философии и социологии \\ 
\end{center}

\vspace{2em}

\begin{center}
\Large \@title \\
(\subtitle)
\end{center}

\vspace{6em}

\begin{flushright}
Выполнил: \\
магистрант 1 курса математического факультета \\
\@author
\end{flushright}


\vspace{\fill}

\begin{center}
Майкоп \\ \yearf\@date
\end{center}

\end{titlepage}
\makeatother


%\maketitle
\tableofcontents

\section{Введение}

Актуальность данного исследования заключается в том, что современная система преподавания математики в некоторой степени излагает математические результаты так, будто они появились сами собой. Возможно, это и хорошо с точки зрения экономии времени и широкого охвата материала, но не менее важно понимать, откуда у математики идут корни. 

Как и любая другая наука, математика создается людьми. И нашей целью будет установить, как мыслили люди, которые делали математику, чтобы ближе соприкоснуться с самим предметом.

Мы рассмотрим программную деятлельность математиков (объект исследования) и их влияние на науку (предмет исследования).

Именно про этих математиков, означенных в теме, можно сказать, что они были \textit{программными} учеными, то есть имели свою \textit{программу}, взгляд на методологию науки и на путь ее развития.

Это ученые, которые основывали научные школы или оказывали значительное влияние на математику в целом. Так, Николай Николаевич Лузин наиболее известен именно созданием научной школы --- так называемой <<лузитании>>.

На Википедии есть фотография древа преемственности, идущее от Лузина, составленное для экспозиции мехмата МГУ. Есть и более глобальный проект, ставящий своей целью собрать генеалогический граф вообще всех существовавших когда-либо математиков \cite{mathgen}.

Для б\'{о}льшего интереса попытаемся составить граф для наших математиков.

\includegraphics{tree}

Если в русской традиции отношение <<учитель-ученик>> довольно определенно, то в западноевропейской оно не так явна. Там, скорее, можно говорить о преемственности или соперничестве.

Например, Анри Пуанкаре довольно жестко критиковал Кантора Пуанкаре называл «канторизм» тяжёлой болезнью, поразившей математическую науку, и выражал надежду, что будущие поколения от неё излечатся \cite{wiki_cantor}. Тогда как Гильберт назвал Кантора «математическим гением» и заявил: «Никто не сможет изгнать нас из рая, созданного Кантором» \cite{wiki_cantor}.

Гильберт же после смерти Пуанкаре перенял эстафету лидерства в мировом математическом сообществе, и считается, наряду с Пуанкаре <<последним математиком-универсалом>>, способным охватить все математические результаты своего времени \cite{wiki_poincare}

\section{Математики и их взгляды на методологию науки}
\subsection{Георг Кантор (1845--1918)}
Георг Кантор оказал значительное влияние на развитие математики.
Кантор внес в математику совершенно новый уровень абстракции.
Он говорил о неопределяемых объектах --- множествах. Множество --- настолько общее понятие, что его элементами может быть что угодно.



Сам Кантор так описал понятие множества: <<Под \textit{множеством} мы понимаем объединение в одно целое определенных, вполне различимых объектов нашей интуиции или нашей мысли>>.

Впрочем, Кантор не был манифестантом, как это происходило сплошь и рядом в науке, а, тем более, и в искусстве.
Кантор был открывателем более, чемe изобретателем. Когда он открыл, что в единичном отрезке и единичном квадрате одинаковое количество точек, он сказал: <<Я вижу это, но никак не могу этому поверить!>>, что подтверждает наш тезис.

Наивный подход к множествам приводил к противоречиям.


$\blacktriangleleft$ Пусть для множества $M$ запись $P(M)$ означает, что $M$ не содержит себя в качестве своего элемента.

Рассмотрим класс $K={M|P(M)}$ множеств, обладающих свойством $P$.

Если $K$ --- множество, то либо верно, что $P(K)$, либо верно, что $\not P(K)$. Но это невозможно. $P(K)$ невозможно из определения $K$; $\not P(K)$ невозможно, так как по определению $K$ тогда было бы верно $P(K)$.

Следовательно, $K$ --- не множество! $\blacktriangleright$

\subsection{Анри Пуанкаре (1854--1912)}

Анри Пуанкаре, возможно, в большей степени из всех был именно философом, поскольку написал довольно объемный труд <<О науке>>.

Противостояние формализма и интуитивизма.

Пуанкаре вообще был против формализма (что следует из его критики Кантора, описанной выше). Но формализм оказал значительное влияние на дальнейшую наукку. В том числе, появление бурбакизма сделало алгебру более абстрактной, и это дало почву для таких наук, как кибернетика, теория алгоритмической сложности, вычислительная математика.

Здравое отношение к проблеме заключается во взвешенном подходе. Если строгость дает выигрыш в технике, то интуиция дает направление мысли. В современной учебной литературе такой подход использовал В. Босс в своей серии <<Интуиция и математика>>.

А. Пуанкаре: <<Я уже имел случай указать то место, какое должна иметь интуиция в преподавании математических наук. Без нее молодые умы не могли бы проникнуться пониманием математики; они не научились бы любить ее и увидели бы в ней лишь пустое словопрение; без нее особенно они никогда не сделались бы способными применять ее.>>\cite{onauke}

CВ математическом доказательстве порядок силлогизмов важнее их содержания. И именно этот порядок определяет ход мысли.

Критика Гильберта за формализм.

\subsection{Давид Гильберт (1862--1943)}

Давид Гильберт, в отличие от Кантора, говорил языком манифеста.

Его программа, его формализм.

Его стремление к аксиоматизации. (Выбрав правильные аксиомы можно построить что угодно).

Его речь на конференции 1900 года, и список глобальных задач.

Правильно заданный вопрос уже содержит в себе часть ответа, поэтому часть проблем Гильберта была решена довольно быстро.

С формализмом приходит некоторое облегчение. Математик теперь не должен думать, чем же <<на самом деле>> является математический объект, если его задают достаточно прозрачне аксиомы.


\subsection{Герман Вейль (1885--1955)}

Герман Вейль также написал книгу о философии математики.

Вейль о Гильберте: 

\subsection{Николай Николаевич Лузин (1883--1950)}

\subsection{Андрей Николаевич Колмогоров (1903--1987)}

\subsection{Владимир Игоревич Арнольд (1937--2010)}

\subsection{Сергей Петрович Новиков (1938)}



\section{Заключение}
Главные выводы

Как выполнены и достигнуты цели

\section{Георг Кантор}

\nocite{*}
\printbibliography


\end{document}
