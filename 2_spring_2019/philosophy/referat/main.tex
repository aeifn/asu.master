% http://coursera.org/course/latex
\documentclass[a4paper,14pt]{extreport}

% Требоване: поля 2,5 см.
\usepackage[left=2.5cm,
	right=2.5cm,
	top=2.5cm,
	bottom=4cm,
	bindingoffset=0cm]{geometry}

% Требование: интервал 1.5
% https://proft.me/2013/06/9/latex-ukazanie-mezhstrochnogo-intervala/ 
\usepackage{setspace}
\onehalfspacing

\usepackage{graphicx}
\graphicspath{ {./images/} }

\usepackage{cmap}					% поиск в PDF
\usepackage{mathtext} 				% русские буквы в фомулах
\usepackage[T2A]{fontenc}			% кодировка
\usepackage[utf8]{inputenc}			% кодировка исходного текста
\usepackage[english,russian]{babel}	% локализация и переносы
\usepackage{hyperref}
\usepackage{amsmath,amsfonts,amssymb,amsthm,mathtools} % AMS

\begin{document}
% https://tex.stackexchange.com/questions/10130/use-the-values-of-title-author-and-date-on-a-custom-title-page
\makeatletter
\begin{titlepage}
\newpage

\begin{center}
Министерство науки и высшего образования РФ \\
Адыгейский государственный университет
\end{center}

\vspace{8em}

\begin{center}
\Large Кафедра философии и социологии \\ 
\end{center}

\vspace{2em}

\begin{center}
\Large \@title \\
(\subtitle)
\end{center}

\vspace{6em}

\begin{flushright}
Выполнил: \\
магистрант 1 курса математического факультета \\
\@author
\end{flushright}


\vspace{\fill}

\begin{center}
Майкоп \\ \yearf\@date
\end{center}

\end{titlepage}
\makeatother


\thispagestyle{empty}
\setcounter{page}{0}
\tableofcontents
\clearpage

\chapter{Философия математики}

Непосредственный предмет математики -- изучение систем математических объектов.

Проблема происхождения этих объектов и их соотношения с объективной реальностью выходит за пределы математики. Тогда применяется философия.

Математические объекты не существуют в объективной реальности. Они --- результат работы человеческого мышления и существуют в сознании человека. 

На этом основании можно охарактеризовать подходы к определению природы математики: эмпиризм, априоризм, формализм.
\cite{mironov}

Математика не относится к естественным, общественным или техничским наукам. Она изучает формы и количественные отношения, одинаково свойственные природе, обществу и человеческому мышлению. Поэтому она является языком науки и формулирует методы научного познания.
\cite{mironov}

Внедрение аксиоматического
метода привело к четырем типам теорий:
\begin{itemize}
\item неаксиматизированные содержательные теории
\item содержательные аксиоматические теории
\item полуформальные аксиоматические теории
\item формальные аксиоматические теории
\end{itemize}

Задача аксиоматических и формальных методов --- обеспечение строгости математического доказательства.

В XIX веке с появлением в математике все более абстрактных понятий и теорий встал вопрос об их обосновании. Стало ясно, что их проверка в естествознании и на практие затруднена или невозможна. 

Обоснование математики приняло форму обоснования непротиворечивости математических теорий. Начался критический пересмотр теорий: от системы аксиом, лежащих в их основе, до правил доказательств и конечных выводов. Первым шагом стала попытка обоснования математики с помощью теории множеств (\ref{cantor}).

Кантор попытался перевести все математические теории на язык теории множеств (все термины и предложения). Для большинства теорий это удалось. Но в самой теории множеств обнаружились логические противоречия, поставившиее под сомнения ее как основание математики.  
\cite{mironov}

Следующим подходом к обоснаванию математики стал \textbf{логицизм}. 
Рассел, Уайтхед, Фреге попытались свести математику к логике.
Логицизм ограничивал идеализацию и запрещал введение объектов, приводящих к парадоксам теории множеств. Но таким образом отбрасывались целые разделы математики, сужался ее предмет. \cite{mironov}

Еще один подход к обоснованию математики --- \textbf{формализм} --- сформировал Давид Гильберт \ref{gilbert}.

Он предлагал формализовать все содержательные математические теории (выделить их форму) и свести обоснование теорий к доказательству непротиворечивости формы. Недостаток этого подхода обнаружился в том, что невозможно полностью формализовать содержательные теории, что доказал Курт Гедель.
\cite{mironov}

Другой подход к основанию математики --- \textbf{интуиционизм} --- вводит критерий интуитивной ясности для оценки математических суждений (Брауэр, Вейль \ref{weyl}, Гейтинг).
В рамках этого подхода ограничивалась идеализация, исключались объекты, требующие более сильной идеализации (например, актуально бесконечное множество), хотя это и сужало предмет математики. \cite{mironov}

До XIX века практическая применимость любых математических теорий казалась нормой. Но с XIX века начали конструироваться все более абстрактные теории. Математические объекты не имели материальных аналогов, однако впоследствии обретали прикладную интерпретацию и наоборот.
С точки зрения философии такая эффективность является подтверждением материального единства мира и принципа детерминизма: за кажущейся хаотичностью скрывается единство мира и его закономерная обусловленность. Разные по природе явления подчиняются сходным количественным закономерностям.
Содержательно разные системы оказываются изоморфными. 
Математические теории фиксируют сходство, и чем более абстрактными они становятся, тем шире могут применяться. 
\cite{mironov}

В XX веке математика 
сыграла роль в становлении неклассического естествознания, в формировании релятивистской и квантовой механики, в исследовании единой теории поля и теории струн.
Математическая красота создаваемых теорий является одним из критериев истинности.


\chapter{Математики и их взгляды на методологию науки}

Это ученые, которые основывали научные школы или оказывали значительное влияние на математику в целом. Так, Николай Николаевич Лузин наиболее известен именно созданием научной школы --- так называемой <<лузитании>>.

\vspace{.5cm}
\includegraphics{tree}

% cantor
\section{Георг Кантор (1845--1918)}
\label{cantor}

Немецкий метематик. Основоположник теории множеств. Работал над обоснованием анализа в смысле Вейерштрасса. Доакзал (1874) несчетность множества всех действительных чисел и тем самым установил существование неэквивалентных бесконечных множеств. Сформулировал (1878) общее понятие множества, первое определение континуума, ввел понятия множест счетных и несчетных, пустых, нулевых, определение границы множества и характеристической функции множества; создал теорию бесконечных и совершенных множеств, теорию трансфинитных кардинальных чисел. Развивал принципы сравнения множеств и доказал эквивалентность множества точек линейного отрезка и точек $n$-мерного многообразия (1878). Систематическое изложение принципов своего учения о бесконечности дал в 1879--1884. Разработал теорию ансамблей. Ввел (1883) новое понятие действительных чисел, которое включило как рациональные действительные, так и иррациональные действительные числа. С 1884 страдал глубокой депрессией и в 1897 отошел от научной деятельности. Основатель и первый президент Германского математического общества (1890--1893), инициатор созыва первого Международного математического конгресса в Цюрихе (1897).
\cite{bogolubov}


Георг Кантор оказал значительное влияние на развитие математики.
Кантор внес в математику совершенно новый уровень абстракции.
Он говорил о неопределяемых объектах --- множествах. Множество --- настолько общее понятие, что его элементами может быть что угодно.


Сам Кантор так описал понятие множества: <<Под \textit{множеством} мы понимаем объединение в одно целое определенных, вполне различимых объектов нашей интуиции или нашей мысли>>.

Впрочем, Кантор не был манифестантом, как это происходило сплошь и рядом в науке, а, тем более, и в искусстве.
Кантор был открывателем более, чемe изобретателем. Когда он открыл, что в единичном отрезке и единичном квадрате одинаковое количество точек, он сказал: <<Я вижу это, но никак не могу этому поверить!>>, что подтверждает наш тезис.

Наивный подход к множествам приводил к противоречиям.

$\blacktriangleleft$ Пусть для множества $M$ запись $P(M)$ означает, что $M$ не содержит себя в качестве своего элемента.

Рассмотрим класс $K={M|P(M)}$ множеств, обладающих свойством $P$.

Если $K$ --- множество, то либо верно, что $P(K)$, либо верно, что $\not P(K)$. Но это невозможно. $P(K)$ невозможно из определения $K$; $\not P(K)$ невозможно, так как по определению $K$ тогда было бы верно $P(K)$.

Следовательно, $K$ --- не множество! $\blacktriangleright$



\section{Давид Гильберт (1862--1943)}\label{gilbert}

Немецкий математик. Основные исследования Гильбарта относятся к теории инвариантов, в которой он сформулировал (1885--1893) основную теорему о существовании конечной базы; алгебраической геометрии, перестроенной (1893-1898) им на основе теории идеалов полиномов; теории алгебраических чисел, где он установил ряд общих законов и, в частности, решил (1909) проблему Варинга относительно возможности разложения любого числа в сумму определенного числа $n$-х степеней целых чисел. Решил (1890--1893) с помощью абстрактных методов основные проблемы теории алгебраических инвариантов. Одним из самы важных направлений в научном творчестве Гильберта были основания геометрии (1898--1902). В книге <<Основания геометрии>> (1899) дал полную систему аксиом евклидовой геометрии. Аксиоматизация, выполненная Гильбертом, была совершенно необходимой в связи с развитием неевклидовых геометрий. Именем Гильберта названо пространство, обобщающее понятие евклидова пространства на бесконечномерный случай (гильбертово пространство). Занимался (1904--1910) теорией интегральных уравнений: построил теорию интегральных уравнений с симметрическим ядром и пришел к ряду понятий, которые легли в основу современного функционального анализа и особенно спектральной теории линейных операторов. Разрабатывал некоторые проблемы анализа, в связи с задачей Дирихле развивал и совершенствовал методы вариационного исчисления. В 1910--1922 обратился к математической физике и вместе с Р. Курантом занимался дальнейшей разработкой и систематизацией ее методов. В 1924 в оавторстве с Курантом опубликовал работу "Методы математической физики". Одновременно интересовался математической логикой, аксиоматизацией арифметики и другими вопросами. Выполнил (1922--1930) важные исследования в области логических оснований математики. Совместно с И. П. Бернайсом написал трактат "Основания математики" (1934).


Давид Гильберт, в отличие от Кантора, говорил языком манифеста.

Его программа олицетворяла тягу к формализму, стремление к аксиоматизации. (Выбрав правильные аксиомы можно построить что угодно).

На конференции 1900 года Гильберт сформулировал список глобальных задач математики.

Правильно заданный вопрос уже содержит в себе часть ответа, поэтому часть проблем Гильберта была решена довольно быстро.

С формализмом приходит некоторое облегчение. Математик теперь не должен думать, чем же <<на самом деле>> является математический объект, если его задают достаточно прозрачне аксиомы.

Гильберт же после смерти Пуанкаре перенял эстафету лидерства в мировом математическом сообществе, и считается, наряду с Пуанкаре <<последним математиком-универсалом>>, способным охватить все математические результаты своего времени \cite{wiki_poincare}

\label{poincare}
\section{Анри Пуанкаре (1854--1912)}

Анри Пуанкаре, возможно, в большей степени из всех был именно философом, поскольку написал довольно объемный труд <<О науке>>.

Противостояние формализма и интуитивизма.

Анри Пуанкаре довольно жестко критиковал Кантора. Пуанкаре называл «канторизм» тяжёлой болезнью, поразившей математическую науку, и выражал надежду, что будущие поколения от неё излечатся \cite{wiki_cantor}. Тогда как Гильберт назвал Кантора «математическим гением» и заявил: «Никто не сможет изгнать нас из рая, созданного Кантором» \cite{wiki_cantor}.

Пуанкаре вообще был против формализма (что следует из его критики Кантора, описанной выше). Но формализм оказал значительное влияние на дальнейшую наукку. В том числе, появление бурбакизма сделало алгебру более абстрактной, и это дало почву для таких наук, как кибернетика, теория алгоритмической сложности, вычислительная математика.

Здравое отношение к проблеме заключается во взвешенном подходе. Если строгость дает выигрыш в технике, то интуиция дает направление мысли. В современной учебной литературе такой подход использовал В. Босс в своей серии <<Интуиция и математика>>.

А. Пуанкаре: <<Я уже имел случай указать то место, какое должна иметь интуиция в преподавании математических наук. Без нее молодые умы не могли бы проникнуться пониманием математики; они не научились бы любить ее и увидели бы в ней лишь пустое словопрение; без нее особенно они никогда не сделались бы способными применять ее.>>\cite{onauke}

В математическом доказательстве порядок силлогизмов важнее их содержания. И именно этот порядок определяет ход мысли.

Критиковал Гильберта за формализм.


\label{weyl}
\section{Герман Вейль (1885--1955)}

Математик и физик.
В области философии математики Вейль примкнул к направлению \textbf{интуиционизма}, по своим взглядам был близок к Пуанкаре \ref{poincare} и Брауэру. Ему принадлежит суждение о наступлении нового кризиса в математике.
Попытка Вейля разработать единую теорию поля потерпела неудачу.
\cite{bogolubov}

Герман Вейль написал книгу о философии математики, в которой осветил важнейшие установки и обобщения философского характера, главным образом выработавшиеся на почве математических и естественно-научных исследований. 

Тщательное установление формальных предпосылок и строгая формулировка издавна назревших проблем является для Вейля делом современности.




\label{luzin}
\section{Николай Николаевич Лузин (1883--1950)}

Советский математик, академик. Ученик Д. Ф. Егорова, вместе с которым основал московскую математическую школу. 
Был выдающимся педагогом. Среди его учеников П. С. Александров, М. А. Лаврентьев, А. Н. Колмогоров \ref{kolmogorov}, Н. К. Бари, П. С. Урысон.



\label{kolmogorov}
\section{Андрей Николаевич Колмогоров (1903--1987)}
Российский ученый, оказавший влияние на развитие ряда разделов математики.

Мировоззрение Колмогорова было последовательно материалистическим. Центральным для него был вопрос о соотношении математических представлений с реальной действительностью. Для философии и методологии математики огромное значение имела статья Колмогорова "Математика" в 1-м (1938) и 2-м (1954) издания БСЭ. Эта статья содержин оригинальную периодизацию математики, анализ предмета и метода математики и ее места в системе наук, а также специальный раздел, посвященный вопросам обоснования математики.
В трудах Колмогорова вскрыты как внешние, так и внутренние мотивы возникновения математических понятий и теорий.

Колмогоров настаивал, что восхождение к более высоким ступеням абстракции имеет прямой практический смысл, и потому предлагал широкое внедрение метода абстракции в предподавание. В 1933 году Колмогоров предложил общепринятую ныне систему аксиоматического обоснования теории вероятностей.

Одной из идей Колмогорова было навести мост между инуиционистской логикой и традиционной, или "классической", логикой, и сделать это средствами, свободными от идеологии интуционизма и от крайностей теоретико-множественного догматизма. Он приводит интерпретацию "классической логики", которая приемлема с точки зрения интуиционизма.

Колмогоров принимает предпринитую главой интуиционизма Брауэром критику традиционной логики. При этом Колмогоров обнаруживает в последней еще один уязвимый логический принцип: $A \to (\neg A \to B)$. Как указывает Колмогоров, эта аксиома "не имеет и не может иметь интуитивных оснований как утверждающая нечто о последствиях невозможного".
\cite{uspensky2002}



\begin{thebibliography}{9}

\bibitem{mironov}
Современные философские проблемы естественных, технических и социально- гуманитарных наук: учебник для аспирантов и соискателей ученой степени кандидата наук / под общ. ред. д-ра филос. наук, проф. В.В.Миронова. – М.: Гардарики, 2007. С. 63-64.

\bibitem{mathgen}
Mathematics Genealogy Project,
Department of Mathematics, North Dakota State University
Fargo, North Dakota
genealogy.math.ndsu.nodak.edu

\bibitem{onauke}
Пуанкаре А.
О науке.
Наука, 1989

\bibitem{vejl}
Герман Вейль
О философии математики
URSS, 2005

\bibitem{bogolubov}
Боголюбов А. Н. Кантор Георг // Математики. Механики. Биографический справочник. — Киев: Наукова думка, 1983. — 639 с.

\bibitem{belyaev}
Беляев Е. А., Киселева Н. А., Перминов В. Я. Некоторые особенности развития математического знания. — Издательство Московского Университета Москва, 1975. — 112 с.

\bibitem{belyaev81}
Беляев, ЕА. Философские и методологические проблемы математики / ЕА. Беляев, В.Я. Перминов. М., 1981.

\bibitem{barabashev97}
Бесконечность в математике: философские и методологические аспекты / под ред. А.Г. Барабашева. М., 1997.

\bibitem{panov87}
Закономерности развития современной математики. Методологические аспекты / отв. ред. М.И. Панов. М., 1987.

\bibitem{kikel89}
Кикель, П.В. Математизация научного знания / П.В. Кикель. Минск, 1989.

\bibitem{perminov02}
Перминов, В.Я. Философия и основания математики / В.Я. Перминов. М., 2002.

\bibitem{rybnikov79}
Рыбников, К А . Введение в методологию математики / К А . Рыбников. М.,
1979.

\bibitem{rybnikov94}
Рыбников, К.А. История математики / КА. Рыбников. М., 1994.

\bibitem{uspensky2002}
В. А. Успенский. Труды по нематематике. В двух томах. М., ОГИ 2002

\bibitem{bell}
Э. Т. Белл. Творцы математики. М., Просвещение, 1979

\end{thebibliography}



\end{document}
