% http://coursera.org/course/latex
\documentclass[a4paper,12pt]{article}


%%% Работа с русским языком
\usepackage{cmap}					% поиск в PDF
\usepackage{mathtext} 				% русские буквы в фомулах
\usepackage[T2A]{fontenc}			% кодировка
\usepackage[utf8]{inputenc}			% кодировка исходного текста
\usepackage[english,russian]{babel}	% локализация и переносы


%%% Дополнительная работа с математикой
\usepackage{amsmath,amsfonts,amssymb,amsthm,mathtools} % AMS
\usepackage{icomma} % "Умная" запятая: $0,2$ --- число, $0, 2$ --- перечисление

%% Номера формул
%\mathtoolsset{showonlyrefs=true} % Показывать номера только у тех формул, на которые есть \eqref{} в тексте.
%\usepackage{leqno} % Немуреация формул слева


\usepackage{hyperref}
\usepackage[usenames,dvipsnames,svgnames,table,rgb]{xcolor}
\hypersetup{				% Гиперссылки
    unicode=true,           % русские буквы в раздела PDF
    pdftitle={Заголовок},   % Заголовок
    pdfauthor={Автор},      % Автор
    pdfsubject={Тема},      % Тема
    pdfcreator={Создатель}, % Создатель
    pdfproducer={Производитель}, % Производитель
    pdfkeywords={keyword1} {key2} {key3}, % Ключевые слова
    colorlinks=true,       	% false: ссылки в рамках; true: цветные ссылки
    linkcolor=red,          % внутренние ссылки
    citecolor=black,        % на библиографию
    filecolor=magenta,      % на файлы
    urlcolor=cyan           % на URL
}

\usepackage{csquotes} % Инструменты для ссылок

\usepackage[backend=biber,bibencoding=utf8,sorting=ynt,maxcitenames=2,style=authoryear]{biblatex}
\addbibresource{sample.bib}


\usepackage{multicol} % Несколько колонок




\author{Егор Кузьмичев, Адыгейский государственный университет}
\title{1.1 Наш первый документ}
\date{\today}

\begin{document} % Конец преамбулы, начало текста.

\maketitle

Задачей этого эссе является познакомиться с алгоритмической теорией игр и ответить на вопросы:
\begin{enumerate}
\item Что такое алгоритмическая теория игр?
\item Какие основные труды существют по АГТ?
\item Какое место АГТ занимает в теории игр и история ее создания
\item Какие применения она имеет?
\item Какие основные теоремы АГТ?
\end{enumerate}

Основные книги для эссе --- \cite{agt}

\tableofcontents

\section{Что такое алгоритмическая теория игр?}

Алгоритмическая теория игр рассматривает задачи, поставленные теорией игр с точки зрения сложности вычислений и других алгоритмических аспектов \parencite{anons12}.
Поэтому преимущественно рассматриваются игры со множеством игроков (тысячи и миллионы). 

\section{Зависимости}

Жесткие: теория вероятностей, теория графов, теория алгоритмов.

Мягкие: теория сложности вычислений, теория игр.

\section{Достижения}

Сложностная классификация задачи о поиске равновесия Нэша.

\section{Основные труды по теории игр}
Теория игр и алгоритмы разрабатывались Джоном фон Нейманом \cite{neumann}.

\section{Глоссарий}

\subsection{Дилемма заключенного (пример с ISP)}

\subsection{The Tragedy of the Commons (<<Трагедия общих ресурсов>>)}
(Например, перегруженный общий прокси-сервер)

\subsection{Pollution game}
случай дилеммы заключенного для многих игроков

\subsection{Coordination Games}
Battle of sexes

\subsection{Равновесие Нэша}

\subsection{Доказательство существования равновесия Нэша в конечной игре}

\subsection{Алгоритмы нахождения равновесия Нэша}

\subsection{Теоретико-сложностный анализ}

\subsection{Игры с полной информацией}

\subsection{Игры с неполной информацией (Байесовские игры)}


\section{Применение}

Моделирование дорожного движения и расчет парковок; также: аукционы, рынки.

\subsection{ISP routing game}
Интересно применение в интернете. Распределение трафика в подсетях своих и чужих.



\printbibliography


\end{document}
