\documentclass[a4paper,12pt]{article}

\usepackage{cmap}					% поиск в PDF
\usepackage[T2A]{fontenc}			% кодировка
\usepackage[utf8]{inputenc}			% кодировка исходного текста
\usepackage[english,russian]{babel}	% локализация и переносы

\author{Егор Кузьмичев}
\title{Лабораторные работы по теории систем}
\date{\today}

\begin{document}

\maketitle

\section{Лабораторная работа \#1}

Вариант 1

\subsection{Может ли какой-нибудь объект или явление быть несистемным?}

Да, объект может быть несистемным. Например, про человека иногда говорят, что он несистемный.

Что такое система? Система --- это полный, целостный набор элементов (компонентов), взаимосвязанных и взаимодействующих между собо так, чтобы могла реализовываться функция системы.

Так что достаточно предположить в некотором наборе компонентов отсутствие взаимосвязей, и этот набор уже будет несистемным.

\subsection{Что такое проблемная ситуация?}
<<Проблемная ситуация>> возникает, если имеется различие между необходимым (желаемым) выходом функции и существующим (реальным).

\subsection{Какие функции выполняют модели во всякой целесообразной деятельности? Можно ли осуществлять такую деятельность без моделирования?}

Модели применяются в том случае, когда необходимо получить информацию об объекте реального мира, абстрагировавшись от второстепенных свойств. Модели позволяют предсказывать и объяснять явления.

Целесообразную деятельность можно осуществлять без моделирования, например, в тех случаях, когда ожидаемая ценность результата такой деятельности меньше, чем затраты на моделирование.

\subsection{Что заставляет пользоваться моделями вместо самих моделируемых объектов?}

Во-первых, в модели используются только важные для данной ситуации аспекты объекта.

Во-вторых, с моделями зачастую проще оперировать.





\section{Лабораторная работа \#2}

Вариант 1

\subsection{Приведите несколько примеров, иллюстрирующих использование свойств естественных объектов в искусственных системах}

Системы делятся на естественные (природные) и искусственные (антропогенные).

В данном случае можно привести пример со станцией выработки электричества, основанной на жизнедеятельности микроорганизмов.

Также можно привести пример с телегой, которую тащит лошадь. Таким образом, в искусственную систему (телега-лошадь) встроен естественный объект (лошадь).

\subsection{Приведите и обсудите свои примеры достижимых и недостижимых целей}

Недостижимы лишь цели, противоречащие объективным природным закономерностям.

\textbf{Недостижимые:} разогнать тело до скорости, превышающей скорость света (если речь не идет о солнечном зайчике); создать вечный двигатель и т. д.

\textbf{Достижимые:} Разработать программный продукт в соответствии с ТЗ, сдать экзамены в университете.

\end{document}
