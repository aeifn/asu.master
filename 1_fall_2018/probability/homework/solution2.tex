\documentclass{article}
\usepackage[utf8]{inputenc}
\usepackage[russian]{babel}
\usepackage{amsmath}
\usepackage{amssymb}
\usepackage{amsfonts}
\usepackage{graphicx}
\usepackage{mathrsfs}
% \usepackage[backend=bibtex8]{biblatex}
% \bibliography{biblio}


\author{Егор Кузьмичёв}


\newcommand{\task}[1]{\vspace{15mm}\textbf{Задача #1. }}

\title{Решение задач по теории вероятностей}



\begin{document}

\maketitle

\tableofcontents
\section{Листок 2}

\task{1}
За круглый стол случайным образом рассаживаются $10$ человек, половина из которых мужчины. 
С какой вероятностью никакие двое мужчин не будут сидеть рядом?

\textbf{Решение.} 
Закон Дирихле. Это возможно, когда все сидят через одного.
Существует ровно одна такая рассадка с точностью до перестановок внутри М и Ж.

Всего рассадок $10!$.

\textbf{Ответ.}
$\frac{5!5!}{10!}$.


\task{2} Коля и Вася договорились встретиться в столовой с 12:00 до 13:00.
Каждый из них приходит в случайное время, Коля ждет ровно 15 минут, затем уходит, а Вася ждет ровно 20 минут, после чего уходит. 
С какой вероятностью они встретятся в столовой?

\textbf{Решение.}
$60^2$ --- всего\\
$60^2-40^2$ --- подходит под условие.

\textbf{Ответ:}
$\frac{60^2-40^2}{60^2}$.


\task{3}
Проводится бесконечная серия испытаний Бернулли с вероятностью успеха 1/3. 
С какой вероятностью первый успех произошел при нечетном испытании?

\textbf{Решение.}

Пространство элементраных исходов описывается так:

$$
\Omega = \{\omega: \omega = (a_1,...), a_i\in\{1,0\}\}
$$
%Успех в серии Бернулли произошел при четном испытании.



%$$\frac{1}{3} + (1-\frac{1}{3})^2 + (1-\frac{1}{3})^2\cdot\frac{1}{3} + ...$$

\task{4}
Докажите, что множество 
$\{\xi^{-1}(B): B\in \mathcal{B}(\mathbb{R})\}$ 
прообразов борелевских множеств под действием случайнов величины $\xi$ является сигма-алгеброй.

\textbf{Решение.} Случайная величина --- это отображние 
$$\xi: \mathscr F\to\mathbb{R}$$
Пусть $(\Omega, \mathscr F)$ --- некоторое измеримое пространство и $(R, \mathscr B(R))$ --- числовая прямая с систомой борелевских множеств $\mathscr B(R)$. 

\textbf{Определение.}
Действительная функция $\xi=\xi(\omega)$, определенная на $(\Omega, \mathscr F)$, называется $\mathscr F$-измеримой функцией, или случайной величиной, если 
$\forall B\in\mathscr B(R) \{\omega:\xi(\omega)\in B\}\in\mathscr F$.
Или, что то же самое, 
$$
\xi^{-1}\equiv\{\omega:\xi(\omega)\in B\}
$$
является измеримым множеством в $\Omega$.


\task{5}
Пусть $\xi$ --- экспоненциальная случайная величина с параметром 2, а $\eta$ равномерно распределена на $[1, 2]$ и не зависит от $\xi$. 
Надите распределение $\eta-\xi$.

\task{6}
Пусть $X$ --- количество $P_3$ (простых путей из двух рбер, т. е. троек вершин, последовательно соединенных ребрами) в $G(n,\frac{1}{n})$. Вычислить $\mathsf{D}X$.

\task{7}
Пусть 
$\xi = \begin{pmatrix}\xi_1\\\xi_2\\\xi_3\end{pmatrix}$ 
	--- гауссовский вектор с вектором средних 
	$\begin{pmatrix}1\\-1\\-2\end{pmatrix}$
		и матрицей ковариаций 
		$\begin{pmatrix}
			3&1&1\\
			1&2&1\\
			1&1&1
		\end{pmatrix}$.
		Найдите характеристическую функцию вектора
		$\begin{pmatrix}
			2&1&-1\\
			-1&2&3\\
			0&-1&-1
		\end{pmatrix}
		\xi +
		\begin{pmatrix}
			1\\
			3\\
			1
		\end{pmatrix}$
		.


\bibliographystyle{unsrt}
\bibliography{biblio}		
\end{document}
