\documentclass{article}
\usepackage[utf8]{inputenc}
\usepackage[russian]{babel}
\usepackage{amsmath}
\usepackage{amssymb}
\usepackage{amsfonts}
\usepackage{graphicx}
\usepackage{mathrsfs}
% \usepackage[backend=bibtex8]{biblatex}
% \bibliography{biblio}


\author{Егор Кузьмичёв}


\newcommand{\task}[1]{\vspace{15mm}\textbf{Задача #1. }}

\title{Решение задач по теории вероятностей}



\begin{document}

\maketitle

\tableofcontents

\section{Листок 1}

\textbf{Задача 1.}
На шахматной доске $8 \times 8$, у которой обрезаны четыре угловых клетки (т. е. всего 60 клеток), случайным образом расположены 8 ладей. Какова вероятность, что никакая пара этих ладей не бьет друг друга?

\textbf{Решение. } 
Переставим нижнюю строку наверх.
Получится в первой стоке $6\cdot5\cdot6!$.
Всего перестановок $\binom{8}{60}$.

Итого --- отношение $$\frac{6\cdot5\cdot6!}{\binom{8}{60}}$$.

\textbf{Задача 2.}
В круглую мишень радиуса 3 наудачу производится три выстрела. Выигрыш при попадании в круг (его центр совпадает с центром мишени) радиуса 1 равен $100$, а при попадании в круг радиуса 2 --- 50 (иначе выигрыш равен 0). Найдите вероятность того, что в результате суммарный выигрыш будет не меньше 200.

\textbf{Решение.}

1. Посчитам вероятности попадания в отдельные круги:\\
\begin{center}
\begin{tabular}{ l l }
	%total area
	$S=\pi R^2$ &  \\
	\hline
	% A_1
	$S_1=\pi$
	& $p_1=\frac{1}{9}$ \\
	\hline
	% A_2
	$S_2=4\pi -\pi=3\pi$ 
	& $p_2=\frac{1}{3}$ \\
	\hline
	% A_3
	$S_3=9\pi-4\pi=5\pi$
	& $p_3=\frac{5}{9}$ \\
	\hline
	%общая площадь
	$S_{all}=9\pi$&a
\end{tabular}
\end{center}

Пространство элементарных событий в одном эксперименте будет: $\Omega = \{\omega_1, \omega_2, \omega_3\}$.

Соответственно, случайные величины будут: 
$$
\eta_1=100, \eta_2=50, \eta_3=0
$$

Теперь перейдем к последовательности трех событий. Всего разных вариантов попадания будет $3^3=27$. 
Каждому такому элементарному событию будет соответствовать случайная величина $\xi: \Omega \to X$,
где $X \in [0,300]$, а именно $X = \{0, 50, 100, 150, 200, 250, 300\}$.
Тогда 
обозначим за $\mathcal{X}$ совокупность всех подмножеств множества $X$.

Нас интересует событие $B\in \mathcal{X}$, заключающееся в том, что $\xi \ge 200$.

Рассмотрим на $(X, \mathcal{X})$ вероятность $P_\xi(\cdot)$, индуцируемую случайной величиной $\xi$ по формуле:

$$
P_\xi(B)=\mathsf{P}\{\omega: \xi(\omega)\in B\}, B\in\mathcal{X}$$.

Ясно, что значения этих вероятностей полностью определяются вероятностями
$$P_\xi(x_i)=\mathsf{P}\{\omega: \xi(\omega)=x_i\}, x_i\in{X}$$\cite[56]{sh}.


\textbf{В целом, ответ такой:} 
$P(x\geqslant200)=3P(1,1,3)+3P(1,2,2)+3P(1,1,2)+P(1,1,1)$.

\textbf{Задача 3.} 
В $100$ испытаниях Бернулли (с вероятностью успеха 1/4 каждое) число успехов равно $X$. Докажите, что вероятность того, что число успехов больше 50 строго меньше, чем 0.5.

\textbf{Решение.}
Случайная величина $\xi$, принимающая два значения $0$ и $1$с вероятностями (<<успеха>>) $p$ и (<<неуспеха>>) $q$ называется \textit{бернулливской}\cite[57]{sh}.
Для нее
$$P_\xi(x)=p^xq^{1-x}, x=0,1.$$\cite[57,1]{sh}

Рассмотрим схему Бернулли:

$$\Omega = \{\omega: \omega=(a_1,..., a_n), a_i=0,1\}, 
p(\omega)=p^{\Sigma{a_i}}q^{n-\Sigma{a_i}}$$
и $\xi_i(\omega)=a_i$ для $\omega=(a_1,...,a_n), i=1,...,n$\cite[59]{sh}.


Вероятность одного успеха --- $S=\frac{1}{4}$.

Матожидание: $\frac{1}{4}+\frac{1}{4}+...=\frac{n}{4}$.

Формула Бернулли:
$$P_n^k=\binom{n}{k}p^kq^{n-k}$$
$$P_n^{k>50}=\Sigma_{k=51}^{100} \binom{100}{k}(\frac{1}{4})^k(\frac{3}{4})^k$$
--- не подходит


Сопоставим успеху $1$, а неуспеху --- $0$. Тогда функция, сопоставляющая строку вида $(1,0,1,1,...)$ числу единиц этой строки будет случайной величиной. 

Сумма может быть одинаковой дл яразных строк, например $(1,0,1)$ и $(1,1,0)$ эквивалентны. 
Число эквивалентных строк определяется количетвом перестановок единиц на возможных местах в строке. 

Всего различных строк возможно $2^{100}$. Вероятность того, что число успеехов строго больше $50$ состоит в отношении количества всех строк с $>50$ единицами к $2^{100}$. 

Однако вероятность возникновения разных строк тоже различна, поскольку $P$ успеха в отдельном испытании равна $\frac{1}{4}$.




\textbf{Задача 4.}
В $100$ испытаниях Бернулли с вероятностью успеха $\frac{1}{4}$ каждое число успехов равно $X$. Докажите, что вероятность того, что число успехов больше $50$ строго меньше, чем $0.03$.

\textbf{Определение.} $
(\Omega,
\mathscr A,
\mathsf P),
\Omega=\{\omega:\omega=(a_1,...,a_n), a_i=0,1\},
\mathscr A={A: A\subseteq\Omega},
\mathsf{P}(\{\omega\})=p^{\Sigma a_i}q^{n-\Sigma a_i} (=p(\omega))$
\cite[69]{sh}.

\textbf{Определение. }
Всякая числовая функция $\xi=\xi(\omega)$, определенная на (конечном) пространстве элементраных событий $\Omega$, называется (простой) случайной величиной.

Пусть $X=\{x_1,...,x_m\}$, где (различными) числами $x_1,...,x_m$ исчерпываются все значения $\xi$.

Обозначим за $\mathscr X$ --- совокупность всех подмножеств множества $X$, и пусть $B\in\mathscr X$. $B$ можно интерпретировать как событие, пространство исходов есть $X$ --- множество значений $\xi$\cite[56]{sh}.


\textbf{Задача 5.}
Проводится бесконечная серия испытаний Бернулли с вероятностью успеха 1/2. Задайте вероятностное пространство и предложите событие, вероятность которого равна 1/3.

\textbf{Решение.}
Число исходов одного эксперимента конечно, число исходов серии --- бесконечно\cite[192]{sh}.

В качестве множества исходов множество:
$$\Omega=\{\omega:\omega=(a_1,a_2,...), a_i=0,1\}$$
Всякое число $a\in[0,1)$
разлагается в двоичную дробь
$$a=\frac{a_1}{2}+\frac{a_2}{2^2}+...(a_i=0,1)$$
между точками $\omega\in\Omega$ и точками $a\in[0,1)$ существует взаимно-однозначное соответствие, значит $|\Omega|=$ континуум.

\textbf{Определение.} Пусть $\Omega$ --- множество точек $\omega$. Система $\mathscr A$ подмножеств $\Omega$ называеся алгеброй, если:\\
a). $\Omega\in\mathscr A$\\
b). $A,B\in\mathscr A\Rightarrow A\cup B\in\mathscr A, A\cap B\in \mathscr A$\\
c). $A\in\mathscr A\Rightarrow\bar A\in\mathscr A$



\textbf{Задача 6.}
На рейс Москва-Пекин выкуплены все билеты. Все пассажиры прибыли на посадку. Первой в самолет заходит старушка, которая садится на случайное место. Пассажиры заходят по почереди, и каждый следующий пассажир занимает своем мест, если оно свободно. Если же место занято, то пассажир садится на случайное место из оставшихся. С какой вероятностью последний пассажир займет свое место?

\textbf{Решение.}
Ответ: $\frac{1}{2}$.

(problems.ru)
Пусть последний сел не на свое место, тогда в тот момент, когда некоторый пассажир занимал место последнего, он мог занять и старушкино место. 

Пусть при некоторой рассаке последний пассажир сел не на свое место (неудачная рассадка), тогда его место было занято пассажиром $A$ (может быть, и старушкой). $A$ выбирал, какое место занять. В рассматриваемом случае он занимал место последнего пассажира. Но с той же вероятностью он мог занять и место старушки, тогда все остальные займут свои места. 

Пока старушкино место вободно, есть ровно 1 пассажир из невошедших, чье место уже занято. Как только пассажир занимает место старушки, все остальные садятся на свои места. Т. е., каждой неудачной рассадке соответствует удачная с той же вероятностью. 


\textbf{Задача 7.}
На схеме изображена электрическая цепь. В элементе с номером $i$ происходит разрыв с вероятностью $\frac{i}{10}$. 
С какой вроятностью элемент с номером $5$ работает, если рабтает вся цепь? 
Цепь работает, если по ней может идти ток. 
Ток идет слва направо. 
Элемент работает, если через него может протечь ток (при подаче тока на вход элемента, а не цепи). 

\textbf{Решение.}
Разрыв в $i$ с $p=\frac{i}{10}$;

$A$ --- элемент $5$ работает.\\
$B$ --- работает вся цепь. 

Цепь работает, если работает $1$ или $3$.
Если работает $1$, то $5$ работает когда работает $4$.

\textbf{Ответ:} $(P(1\cup4)+P(3\cup2))P(5)$.


8. Пусть $\mathcal{F}_1$ и $\mathcal{F}_2$ --- две $\sigma$-алгебры подмножеств $\Omega$. Верно ли, что $\mathcal{F}_1\cap\mathcal{F}_2$ является $\sigma$-алгеброй? А $\mathcal{F}_1\cup\mathcal{F}_2$?

\textbf{Определение.}
Система $\mathscr F$ подмножеств $\Omega$ называется $\sigma$-алгеброй, если она является алгеброй и, кроме того, выполнено свойство \\
b*) если $A_n\in\mathscr F, n=1,2,...$, то $\bigcup A_n\in\mathscr F, \bigcap A_n\in\mathscr F$.

a) $\Omega\in\mathscr F$\\
c) $A\in\mathscr F\Rightarrow \bar A\in\mathscr F$

\textbf{Ответ:}
$\mathscr F_1\cup F_2$ не является $\sigma$-алгеброй, т. к. не выполнено a).\\
$\mathscr F_2\cap\mathscr F_2$ является $\sigma$-алгеброй, т.к выполнено a), b*), c).

\textbf{Задача 9.} 
Пусть $\xi_1,...,\xi_n$ --- независимые стандартные нормальные случайные величины. Найдите распределение величины $\xi_1^2+...+\xi_n^2$.

\textbf{Решение.}
Рассмотрим вопрос об отыскании распределения суммы случайных величин $\zeta=\xi+\eta$.

Пусть 
$(\Omega, \mathscr F)$ 
--- некоторое измеримое пространство и 
$(R, \mathscr B(R))$ 
--- числовая прямая с системой борелевских множеств $\mathscr B(R)$.

\textbf{Определение.} 
Действительная функция $\xi=\xi(\omega)$, определенная на $(\Omega, \mathscr F)$ называется $\mathscr F$-измеримой функцией или случайной величиной, если 
$\forall B\in\mathscr B(R) \{\omega: \xi(\omega)\in B\}\in\mathscr F$.

\textbf{Решение.}
Стандартное нормальное распределение --- функция, симметричная относительно оси $O_y$, значит сумма двух таких функций будет обладать теми же свойствами.

\textbf{Задача 11.}
Пусть $\xi$ --- положительная абсолютно непрерывная случайная величина, $a<b$ --- произвольные положительные числа. Докажите, что если существует конечное $\mathsf{E}\xi^b$, то и конечно $\mathsf{E}\xi^a$.


12. Докажите, что если последовательность случайных велчичин $\xi_1, \xi_2, ...$ сходится к константе $c\in\mathbb{R}$ по распределению, то тогда и $\xi_n\xrightarrow{P}c$ (есть сходимость по вероятности к той же константе).

13. Пусть $\xi$ --- пуассоновская случайная величина с параметром $2$. Найдите $\mathsf{E}\xi^4$. 

14. Найдите характеристическую функцию экспоненциального распределения с параметром 3.

15. 
Пусть 
$\xi = \begin{pmatrix}\xi_1\\\xi_2\end{pmatrix}$
	--- гауссовский вектор с вектором средних 
	$\begin{pmatrix}1\\-1\end{pmatrix}$
		и матрицей ковариаций 
		$\begin{pmatrix}2&1\\1&3\end{pmatrix}$. 
Найти вектор средних и матрицу ковариаций вектора 
$\begin{pmatrix}1&0\\-2&1\end{pmatrix}\xi+\begin{pmatrix}2\\-3\end{pmatrix}$.


\bibliographystyle{unsrt}
\bibliography{biblio}		
\end{document}
