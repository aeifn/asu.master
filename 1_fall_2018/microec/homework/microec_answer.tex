\documentclass[a4paper,12pt]{article}

\usepackage{cmap}					% поиск в PDF
\usepackage[T2A]{fontenc}			% кодировка
\usepackage[utf8]{inputenc}			% кодировка исходного текста
\usepackage[english,russian]{babel}	% локализация и переносы

\author{Егор Кузьмичев}
\title{Микроэкономика}
\date{\today}

\begin{document} % Конец преамбулы, начало текста.

\maketitle

\section{Задание 1}
Спрос и предложение заданы функциями $q_D=a-bp$ и $q_S=cp-d$, $a,b,c,d>0$ --- заданные самостоятельно коэффициенты.

1). Нати точку равновесия $p^*, q^*$.

Это будет точка пересечения графиков $S, D$.
А именно, точка $x$: $a-bx=cx-d$.
$(c+b)x=a+d; x=\frac{a+d}{c+b}$. 

2). При какой цене будет максимальная выручка?

Выручка $TR:=pq$.

Надо: $TR\to\max$

Будем считать по спросу: $TR=pq=p(a-bp)=pa-p^2b$.

$TR'=a-2pb$

$0=a-2pb; 2pb=a; p=\frac{a}{2b}$


3). Что произойдет с точкой равновесия, если спрос вырастет на $30\%$, а предложение снизится на $20\%$?

Равновесная цена увеличится, и равновесное количество тоже увеличится.

4). Что произойдет с точкой равновесия, если государство введет налог в размере t руб. за единицу продукции?

Изменится функция предложения.

Она будет иметь вид:
$q_S=c(p-t)-d$. А дальше ищется равновесие, как в п. 1.
Цена вырастет, предложение сократится.

5). Найти ценовую эластичность спроса и предложение в точке равновесия.

$E=\alpha\frac{p^*}{q^*}$, где $\alpha$ --- производная графиков спроса или предложения.

6). Взять некоторую цену $p>p^*$. Найти потребительский излишек, излишек производителя и общественное благосостояние при данной цене.

Оба излишка будут отрицательными.
Находятся они так:

$$
PS=\int_{p^*}^p p_S dp - \Delta p \cdot q
$$
$$
CS=-\int_{p^*}^p q_S dp + \Delta p \cdot q
$$
$$
SW=CS+PS
$$

7). Найти оптимальный объем производства монополии, работающей на рынке с указанным спросом, если ее издержки производства имеют вид:

$$
TC=\tilde f + \tilde cq + \tilde dq^2, \tilde c = \frac{d}{c}, \tilde d=\frac{1}{2c}.
$$

Условие максимизации прибыли:
$MC=MR$

$MC=TC'=\tilde c + 2\tilde dq$

$MR=TR'=(pq)'=\frac{a-2q}{b}$

Остается только приравнять в найти q в линейном уравнении с 1 неизвестным. 




\end{document} % Конец текста.
