\documentclass{article}
\usepackage[utf8]{inputenc}
\usepackage[russian]{babel}
\usepackage{amsmath}
\usepackage{amsfonts}
\usepackage{graphicx}
%\usepackage{sansmath}
\usepackage[backend=bibtex8]{biblatex}
\bibliography{biblio}

\title{Реферат по теории сложности вычислений}
\author{Егор Кузьмичёв}


\begin{document}

\maketitle

\tableofcontents

\section{Машина Тьюринга}
Что может делать машина Тьюринга и что на ней можно посчитать?

Во-первых, это бесконечная в обе стороны лента, разделенная на ячейки.

И есть управляющее устройство с конечным числом состояний. 
Это управляющее устройство может перемещаться влево и вправо по ленте, читать и записывать в ячейки некоторого конечного алфавита.

Что такое конечный алфавит, вполне понятно. Можно представить, что, например, алфавит состоит из букв {0,1}. 
Бесконечную в обе стороны ленту тоже представить легко. Достаточно подумать о множестве /Z целых чисел.
По сути, эта машина - мысленный эксперимент, а не устройство. 
Может быть, поэтому приобрести такую машину можно дешевле, чем купить материальный компьютер, а работает она гораздо быстрее, если не сказать моментально.

%Как говорил математик Владимир Арнольд, "разница между математикой и физикой состоит только в том, что в физике эксперименты стоят миллионы или даже миллиарды долларов, а в математике — единицы рублей или копеек".

Тьюринг придумал машину для \textit{формализации понятия} алгоритма.

Машина Тьюринга является расширением модели конечного автомата (что находит, видимо, свое отражение в том, что у управляющего устройства конечное число состояний).

Конечный автомат может быть задан в виде упорядоченной пятерки множеств:

$M = (V, Q, q_0, F, \delta)$, где\\
$V$ --- входной алфавит (конечное множество входных символов), из которого формируются входные слова, воспринимаемые конечным автоматом,\\
$Q$ --- множество внутренних состояний,\\
$q_0$ --- начальное состояние ($q_0 \in Q);\\
$\\
$\delta$ --- функция переходов, определнная как отображение $\delta:Q\times(V\cup\{\epsilon\})\to Q$, такое, что $\delta(q,a)=\{r:q\underset{a}{\to} r\}$, то есть значение функции переходов на упорядоченной паре есть множество всех состояний, в которые из данного состояния возможен переход по данному входному символу или пустой цепочке

У машины есть пустой символ, который содержится во всех клетках, где не зависаны данные.

В управляющем устройстве содержится таблица переходов., которая представляет алгоритм, реализуемый данной машиной.

Каждое правило из таблицы предписывает машине, в зависимости от текущего состояния, и символа в текущей клетке, записать в эту клетку новый символ, перейти в новое состояние и переместиться на одну клетку влево или вправо.

Некоторые состояния помечены как \textit{терминальные}. 

Автомат начинает работу в состоянии $q_0$, считывая по одному символы входной строки. считанный символ переводит автомат в новое состояние из $Q$. 

Для машины Тьюринга правило перехода предписывает, в зависимости от текущего состояния, записать в текущую клетку символ алфавита, перейти в новое состояние и на одну клетку влево или вправо.

Более точно, конкретная машина Тьюринга имеет набор правил вида: 
${q_i}{a_j}\to {q_{i1}}{a_{j1}}{d_k}$, 
где $q$ --- состояние, $a$ --- буква, d --- сдвиг на $1$ или $0$.

В теории сложности вычислений как раз логично взять за единицу выполнение такого правила.
\cite{wikituring}.

\section{Сложность задач}
Моор и Мертенс в книге <<Природа вычислений>>\cite{moor} начинает повествование с Эйлера и Кенигсбергского моста.
И вообще говоря, эта задача относится к разряду комбинаторных. Как написал сам Эйлер, <<задача может быть решена исчерпывающим списком возможных путей, и тогда можно найти из них те, которые удовлетворяют условию задачи. Но поскольку количество путей велико, этот метод будет слишком сложным и трудным, а если прибавить еще мостов, то и невозможным>>.

Таким образом, становится понятно, чем вообще занимается теория сложности вычислений. Снижением алгоритмической сложности решения задачи. Кроме того, теория рассматривает не отдельные задачи, а семейства схожих задач, и то, как растет, или масштабируется, их сложность  в зависимости от количества ребер графа.

Интересно, что в рамках ТСВ, множество проблем --- поиск Гамильтонова пути, раскраска графов и др. --- одинаково сложные. И если есть возможность решить эффективно одну из них, то можно решить и все. Но тогда надо понять, что общего в них и как трансформировать одну задачу в другую?

Так, например, если найти эффективный способ поиска Гамильтоновых путей, то недалеко и до решения гипотезы Римана.

Еще интересный вопрос - может ли один язык программирования, или какой-то один тип компьютера решать такие задачи, который другие не могут? Или же все достаточно мощные компьютеры одинаковы? 

Существуют ли задачи, с которыми никакой копьютер не сможет справиться за любое время?

Как соотносятся количество памяти и время выполнения? Сколько памяти нужно, чтобы выбраться из лабиринта или найти выигрышную стратегию в игре с двумя игроками?

Еще интересный вопрос. Если у Мерлина в голове суперкомпьютер, а Артур простой смертный, может ли Мерлин убедить Артура, что выиграет у него в шахматы, даже не играя?

Так вот, это все можно формализовать при помощи алгоритма.

%Как сказал Дональд Кнут, "говорят, что чтобы понять что-то, надо этому научить другого. Но на самом деле, для полного понимания достаточно научить компьютер... Попытка формализовать решение в виде алгоритма приводит к более глубокому пониманию вещей."

Как можно улучшить алгоритм, чтобы избавиться от исчерпывающего (exhaustive) перебора? Нужно предположить, что путь уже кому-то известен, и тогда останется только проверить, действительно ли он соответствует условию. Так возникает идея верификатора, который должен проверить соответствие пути условию.

Классическая задача про иголку в стоге сена. Если стог экспоненциального размера, то задача трудна.  
Но указание на иголку является непосредственным доказательством ее существования, однако не факт, что ее можно легко найти перебором. 


\section{Классы сложности вычислений $P$ и $NP$}

Это как раз и определяет класс $NP$ (от англ. non-deterministic polynomial). Обобщенно можно сказать, это класс задач, решение которых легко проверить, но не обязательно легко найти это решение.

В то время как класс $P$ --- это задачи, решение которых можно найти за время, описываемое полиномом (время в данном случае --- это количество итераций машины Тьюринга).

Интересно заметить, что теория сложности вычислений вообще не решает никакой конструктивной задачи, а только теоретические. То есть она не отвечает на вопрос, как что-то сделать, а, скорее, а вопрос возможно ли это сделать вообще и с какой сложностью.

Вообще говоря, вся идея сложности вычислений вращается вокруг метафоры стога сена и иголки.

%Например, Гамильтонов путь. 
Это задача принятия решения (decision problem), с бинарным ответом (да/нет).
Если ответ да - это легко проверяемо, достаточно взять сам путь.

Класс $NP$ --- класс, в котором легко проверить правильность ответа. Под простой проверкой имеется в виду возможность проверки за полиномиальное время.

Как и гамильтонов путь, многие проблемы из NP имеют форму <<существует ли такой способ, что...>>. Если существует, то доказательством этого факта является сам способ.

Большинство задач, интересных с практической точки зрения, имеют полиномиальные (работающие за полиномиальное время) алгоритмы решения. То есть время работы алгоритма на входе длины n составляет не более O(nk) для некоторой константы k (не зависящей от длины входа)\cite{rain}.

Если мы хотим провести пусть грубую, но формальную границу между «практическими» алгоритмами и алгоритмами, представляющими лишь теоретический интерес, то класс алгоритмов, работающих за полиномиальное время, является разумным первым приближением.
Мы рассмотрим, руководствуясь, класс задач, называемых $NP$-полными.
Для этих задач не найдены полиномиальные алгоритмы, однако и не доказано, что таких алгоритмов не существует.
Изучение NP-полных задач связано с так называемым вопросом <<$P = NP$>>\cite{rain}.

\subsection{История вопроса}

Концепция $NP$-полноты была введена в 1971 году (см. Cook-Levin theorem), хотя сам термин <<$NP$-полнота>> был придуман позже. Ученых интересовал вопрос, могут ли $NP$-полные задачи быть решены за полиномиальное время на машине Тьюринга (determenistic). Однако этот вопрос был отложен на будущее и до сих пор не решен. 
\cite{wikinpcompleteness}.

\subsection{Что, если $P=NP$}
$P$ обозначает задачи, которые могут быть легко решены при помощи компьютеров. $NP$ относится к проблемам, к котором хотелось бы найти лучшее решение. Если $P=NP$, то тогда можно легко найти решение к любой задаче. Если $P=NP$, тогда изменится жизнь общества, включая незамедлительные улучшения в медицине, науке, развлечениях и автоматизации почти любой человеческой деятельности.
$P=NP$ значит, что нет универсального способа решить некоторые задачи.
\cite{goldenticket}

\section{Формальные определения}
\subsection{Класс $P$}
Классом $P$ называется множество языков, распознающихся за полиномиальное время. Он определяется как:
$$
\mathsf{P}=\bigcup_{c=1}^\infty \mathsf{DTIME}(n^c)
$$ 
\cite{musatov}

\subsection{Класс $NP$}
Класс сложности $NP$ может быть определен как
$$
\mathsf{NP} = \bigcup_{k\in\mathbb{N}} \mathsf{NTIME}(n^k)
$$
где
$\mathsf{NTIME}(n^k)$ --- множество задач принятия решения, которые могут быть решены на недетерменированной машине Тьюринга за время $O(n^k)$\cite{wikiclassnp}.

%\medskip
\printbibliography
%\bibliographystyle{IEEEtran}
%\bibliography{biblio}
\end{document}
