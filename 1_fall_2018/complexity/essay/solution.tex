\documentclass[a4paper,12pt]{article}

\usepackage{cmap}					% поиск в PDF
\usepackage[T2A]{fontenc}			% кодировка
\usepackage[utf8]{inputenc}			% кодировка исходного текста
\usepackage[english,russian]{babel}	% локализация и переносы
\usepackage{amsmath}

\author{Егор Кузьмичёв}
\title{Решение задач по теории сложности вычислений}
\date{\today}

\begin{document} % Конец преамбулы, начало текста.

\maketitle


\section{Придумайте $NP$-полные языки $A$ и $B$, такие что $A\cap B$ и $A\textbackslash B$ также $NP$-полные.}

\subsection{NP-полные задачи}

NP-полная задача --- это задача с ответом <<да>> или <<нет>>, из класса NP, к которой можно свести любую другую задачу из этого класса за полиномиальное время (то есть при помощи операций, число которых не превышает некоторого полинома в зависимости от размера исходных данных).

% "как сводить одну задачу к другой".

\subsection{Язык}

Алфавитом называется всякое конечное множество символов (например, \{0,1\} или \{a,b,c\}). Множество всех возможных слов (конечных строк, составленных из этого алфавита) над некоторым алфавитом обозначается $\Sigma^*$. Языком $L$ над алфавитом $\Sigma$ называется всякое подмножество множества $\Sigma^*$, то есть $L\subset \Sigma^*$.

\subsection{NP-полный язык}
Задачей распознавания для языка $L$ называется определение того, принадлежит ли данное слово языку $L$.

Пусть $L_1$ и $L_2$ --- два языка над алфавитом $\Sigma$. Язык $L_1$ называется сводимым к языку $L_2$, если существует функция $f: \Sigma^*\to \Sigma^*$, вычислимая за полиномиальное время, обладающая следующим свойством:
$x\in L_1 \Leftrightarrow f(x)\in L_2$.

Язык $L_2$ называется NP-трудным, если любой язык из класса NP сводится к нему. Язык называют NP-полным, если он NP-труден, и при этом сам лежит в классе NP.

\section{Докажите NP-полноту языка SUBGRAPH-ISOMORPHISM = \{(G, H) | в G есть подграф (необязательно индуцированный), изоморфный H\}}

(по материалу Википедии)

Для доказательства, что задача поиска изоморфного подграфа NP-полна, её нужно сформулировать как задачу разрешимости.  Входом задачи разрешимости служит пара графов ''G'' и ''H''. Ответ задачи положителен, если ''H'' изоморфен некоторому подграфу графа ''G'', и отрицателен в ином случае.

Формальное задание:

Пусть $G=(V,E)$, $H=(V^\prime,E^\prime)$ —  два графа. Существует ли подграф $G_0=(V_0,E_0): V_0\subseteq V, E_0\subseteq E\cap(V_0\times V_0)$, такой, что $G_0\cong H$? Т.е. существует ли отображение $f\colon V_0\rightarrow V^\prime$, такое, что $(v_1,v_2)\in E_0\Leftrightarrow (f(v_1),f(v_2))\in E^\prime$?

Доказательство NP-полноты задачи поиска изоморфного подграфа просто и основывается на сведении к этой задаче задачи о клике, NP-полной задачи разрешимости, в которой входом служит один граф ''G'' и число ''k'', а вопрос состоит в следующем: содержит ли граф ''G'' полный подграф с ''k'' вершинами. Для сведения этой задачи к задаче поиска изоморфного подграфа, просто возьмём в качестве графа ''H'' полный граф $K_k$. Тогда ответ для задачи поиска изоморфного подграфа с входными графами  ''G'' и ''H'' равен ответу для задачи о клике для графа ''G'' и числа ''k''. Поскольку задача о клике NP-полна, такое сведение полиномиального времени показывает, что задача поиска изоморфного подграфа также NP-полна.

Альтернативное сведение от задачи о гамильтоновом цикле отображает граф ''G'', который проверяется на гамильтоновость, на пару графов ''G'' и ''H'', где ''H'' — цикл, имеющий то же число вершин, что и ''G''. Поскольку задача о гамильтоновом цикле является NP-полной даже для планарных графов, это показывает, что задача поиска изоморфного подграфа остаётся NP-полной даже для планарного случая.

Задача поиска изоморфного подграфа является обобщением задачи об изоморфизме графов, которая спрашивает, изоморфен ли граф ''G'' графу ''H'' — ответ для задачи об изоморфизме графов «да» тогда и только тогда, когда графы ''G'' и ''H'' имеют одно и то же число вершин и рёбер и задача поиска изоморфного подграфа для графов  ''G'' и ''H'' даёт «да». Однако статус задачи изоморфизма графов с точки зрения теории сложности остаётся открытым.

Грёгер показал, что если выполнена гипотеза  Аандераа – Карпа – Розенберга о сложности запроса монотонных свойств графа, то любая задача поиска изоморфного подграфа имеет сложность запроса $\Omega$ ($n_{3/2}$). То есть решение задачи поиска изоморфного подграфа требует проверки существования или отсутствия во входе $\Omega$ ($n_{3/2}$) различных рёбер графа.
\end{document}
