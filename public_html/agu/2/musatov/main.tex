\documentclass{article}
\usepackage[utf8]{inputenc}
\usepackage[russian]{babel}

\title{ЗАДАЧКИ ПО МАТЕМАТИКЕ ДЛЯ ГУМАНИТАРИЕВ}
\date{October 2017}

\usepackage{natbib}
\usepackage{graphicx}

\begin{document}

\maketitle

\indent 1. Постройте с помощью циркуля и линейки правильный пятиугольник (трудная!).

2. Упростите выражение, убрав иррациональность из знаменателя:
$$
\frac{1}{3+\sqrt{2}+\sqrt{3}}
$$

3. Разделите с остатком многочлен $5x^6 + 7x^4 - 3$ на $x^4 - x^3 + 1$.

4. С помощью формулы Кардано решите уравнение $x^3 - 6x - 6 = 0$.

5. Найдите наибольший общий делитель чисел $34$ и $52$, и представьте
его в форме $34n + 52m$, где $n$ и $m$ должны быть целыми числами.

6. Найдите множество всевозможных сумм чисел, лежащих на отрезках
$[-2,-3]$ и $[5,6]$. (Называется "суммой Минковского")

7. Установитевзаимно-однозначное соответствие между множеством
точек отрезка $[0,1]$ и интервала $(0,1)$. (трудная!)

8. Разложите в цепную дробь $\sqrt{7}$.

9. Найдите многочлен с целыми коэффициентами, минимальный по степени из
всех, имеющих корнем число $\sqrt{2} + \sqrt{7}$.

10. Постройте с помощью циркуля и линейки число $\sqrt{3+\sqrt{7}}$.

11. Докажите неразложимость многочлена $x^5 - 3x + 3$ на множители,
являющиеся многочленами с целыми коэффициентами (из чего следует
по лемме Гаусса и неразложимость над полем рациональных чисел).

12. Опишите поле, являющееся минимальным полем, содержащим все
рациональные числа (дроби), а также конкретное число $\sqrt{5} + \sqrt{11}$.

13. Разделите в этом поле число $2+\sqrt{5}$ на $1 + \sqrt{5} + \sqrt{7}$.

\end{document}
