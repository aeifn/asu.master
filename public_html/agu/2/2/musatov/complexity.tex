\documentclass{article}
\usepackage[utf8]{inputenc}
\usepackage[russian]{babel}
\usepackage{amsmath}
\usepackage{natbib}
\usepackage{graphicx}

\title{Эссе по теории сложности вычислений}
\author{Егор Кузьмичёв}

\begin{document}

\maketitle
Мне всегда были интересны компьютеры и программы, которые на них запускаются.
Но при этом я никогда не понимал, в чем заключается прелесть математики.
Программы были для меня скорее средством взаимодействия с другими людьми, средством коммуникации.
В том смысле, что делая программы, я пытался сделать что-то полезное для общества, получить какое-то призвание.

Однажды, еще в 8-9 классах, я написал программу для настройки сетевого интерфейса районному провайдеру (клиент ставил программу и монтажнику не приходилось вручную настраивать компьютер).

Были у меня также программы, например, текстовый редактор DPad или вообще даже программа, которая подсчтиывает расстояние, пройденное курсором мыши.

Мы назвали это YSsoft, потому что там был я (yegor) и мой одноклассник (slava).

Мы даже были на конкурсе программистов, организованном Интелом. Но там я не смог ответить на вопрос про то, как соотносятя количества пикселей с расстояниями, которое проходит курсор и мышь и об устройстве дисплея.

В школе я также написал программу для тестирования по сети. 

Жаль, что я только уже во взрослом возрасте познакомился с UNIX-системами и подходом с текстовыми потоками, воплощенном в них.

Мне кажется, что это гораздо ближе к тому, что называется Computation. 

Однако, надо заметить, что занятия компьютерными науками возможны и совсем без компьютера. Ведь это же чистая математика. 

Вместо компьютера можно использовать математическую модель, например, машину Тьюринга. 
Тьюринг, кстати, видимо был довольно честолюбивым, поскольку, как сказал Александр Шень, "Тьюринг придумал машину, которую назвал машиной Тьюринга".
Правда это или нет, я точно не знаю.

Тем не менее, для меня занятия каким-то предметом обуславливаются интересом к нему. Мне интересно отвечать на собственные вопросы, но не так интересно выполнять задание, данное кем-то. 

Ну вот начать с машины Тьюринга. Что она может сделать и что на ней можно посчитать?

Ну во-первых, это бесконечная в обе стороны лента, разделенная на ячейки.

И есть управляющее устройство с конечным числом состояний. 
Это управляющее устройство может перемещаться влево и вправо по ленте, читать и записывать в ячейки некоторого конечного алфавита.

Что такое конечный алфавит, вполне понятно. Можно представить, что, например, алфавит состоит из букв {0,1}. 
Бесконечную в обе стороны ленту тоже представить легко. Достаточно подумать о множестве /Z целых чисел.
По сути, эта машина - мысленный эксперимент, а не устройство. 
Может быть, поэтому приобрести такую машину можно дешевле, чем купить материальный компьютер, а работает она гораздо быстрее, если не сказать моментально.

Как говорил математик Владимир Арнольд, "разница между математикой и физикой состоит только в том, что в физике эксперименты стоят миллионы или даже миллиарды долларов, а в математике — единицы рублей или копеек".

Несколько фактов:

Оказывается, Тьюринг придумал машину для \textit{формализации понятия} алгоритма.

Машина Тьюринга является расширением модели конечного автомата (что находит, видимо, свое отражение в том, что у управляющего устройства конечное число состояний).

И вот тут интересно узнать про управляющее устройство, а точнее про конечный автомат. Что значит, что число возможных внутренних состояний конечно?

Ну чтобы быть конеретным, скажем, что конечный автомат может быть задан в виде упорядоченной пятерки множеств:

$M = (V, Q, q_0, F, \delta)$, где\\
$V$ --- входной алфавит (конечное множество входных символов), из которого формируются входные слова, воспринимаемые конечным автоматом,\\
$Q$ --- множество внутренних состояний,\\
$q_0$ --- начальное состояние ($q_0 \in Q);\\
$\\
$\delta$ --- функция переходов, определнная как отображение $\delta:Q\times(V\cup\{\epsilon\})\to Q$, такое, что $\delta(q,a)=\{r:q\underset{a}{\to} r\}$, то есть значение функции переходов на упорядоченной паре есть множество всех состояний, в которые из данного состояния возможен переход по данному входному символу или пустой цепочке.

У машины есть пустой символ, который содержится во всех клетках, где не зависаны данные.

В управляющем устройстве содержится таблица переходов., которая представляет алгоритм, реализуемый данной машиной.

Каждое правило из таблицы предписывает машине, в зависимости от текущего состояния, и символа в текущей клетке, записать в эту клетку новый символ, перейти в новое состояние и переместиться на одну клетку влево или вправо.

Некоторые состояния помечены как \textit{терминальные}. 

Автомат начинает работу в состоянии $q_0$, считывая по одному символы входной строки. считанный символ переводит автомат в новое состояние из $Q$. 

Для машины Тьюринга правило перехода предписывает, в зависимости от текущего состояния, записать в текущую клетку символ алфавита, перейти в новое состояние и на одну клетку влево или вправо.

Более точно, конкретная машина Тьюринга имеет набор правил вида: ${q_i}{a_j}\to {q_i_1}{a_j_1}{d_k}$, где $q$ --- состояние, $a$ --- буква, d --- сдвиг на $1$ или $0$.

В теории сложности вычислений как раз логично взять за единицу выполнение такого правила.

Посмотрим, с чего начинает свое повествование Моор и Мертенс в книге <<Природа вычислений>>? С Эйлера и Кенигсбергского моста. 

И вообще говоря, эта задача относится к разряду комбинаторных. Как написал сам Эйлер, "задача может быть решена исчерпывающим списком возможных путей, и тогда можно найти из них те, которые удовлетворяют условию задачи. Но поскольку количество путей велико, этот метод будет слишком сложным и трудным, а если прибавить еще мостов, то и невозможным".

Таким образом, становится понятно, чем вообще занимается теория сложности вычислений. Снижением алгоритмической сложности решения задачи. Кроме того, теория рассматривает не отдельные задачи, а семейства схожих задач, и то, как растет, или масштабируется, их сложность  в зависимости от количества ребер графа.







https://ru.wikibooks.org/wiki/Машина_Тьюринга\\
https://ru.wikipedia.org/wiki/Конечный_автомат\\
The Nature of Computation Cristopher Moore, Stephan Mertens

\end{document}
