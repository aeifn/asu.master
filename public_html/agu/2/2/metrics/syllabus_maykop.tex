\documentclass[11pt,russian,]{article}
\usepackage{lmodern}
\usepackage{amssymb,amsmath}
\usepackage{ifxetex,ifluatex}
\usepackage{fixltx2e} % provides \textsubscript
\ifnum 0\ifxetex 1\fi\ifluatex 1\fi=0 % if pdftex
  \usepackage[T1]{fontenc}
  \usepackage[utf8]{inputenc}
\else % if luatex or xelatex
  \ifxetex
    \usepackage{mathspec}
  \else
    \usepackage{fontspec}
  \fi
  \defaultfontfeatures{Ligatures=TeX,Scale=MatchLowercase}
    \setmainfont[]{Arial}
\fi
% use upquote if available, for straight quotes in verbatim environments
\IfFileExists{upquote.sty}{\usepackage{upquote}}{}
% use microtype if available
\IfFileExists{microtype.sty}{%
\usepackage{microtype}
\UseMicrotypeSet[protrusion]{basicmath} % disable protrusion for tt fonts
}{}
\usepackage[left=2cm, right=2cm, top=2cm, bottom=2cm]{geometry}
\usepackage{hyperref}
\PassOptionsToPackage{usenames,dvipsnames}{color} % color is loaded by hyperref
\hypersetup{unicode=true,
            pdftitle={Программа по эконометрике},
            pdfauthor={Борис Демешев},
            colorlinks=true,
            linkcolor=blue,
            citecolor=blue,
            urlcolor=blue,
            breaklinks=true}
\urlstyle{same}  % don't use monospace font for urls
\ifnum 0\ifxetex 1\fi\ifluatex 1\fi=0 % if pdftex
  \usepackage[shorthands=off,main=russian]{babel}
\else
  \usepackage{polyglossia}
  \setmainlanguage[]{russian}
\fi
\usepackage{longtable,booktabs}
\usepackage{graphicx,grffile}
\makeatletter
\def\maxwidth{\ifdim\Gin@nat@width>\linewidth\linewidth\else\Gin@nat@width\fi}
\def\maxheight{\ifdim\Gin@nat@height>\textheight\textheight\else\Gin@nat@height\fi}
\makeatother
% Scale images if necessary, so that they will not overflow the page
% margins by default, and it is still possible to overwrite the defaults
% using explicit options in \includegraphics[width, height, ...]{}
\setkeys{Gin}{width=\maxwidth,height=\maxheight,keepaspectratio}
\IfFileExists{parskip.sty}{%
\usepackage{parskip}
}{% else
\setlength{\parindent}{0pt}
\setlength{\parskip}{6pt plus 2pt minus 1pt}
}
\setlength{\emergencystretch}{3em}  % prevent overfull lines
\providecommand{\tightlist}{%
  \setlength{\itemsep}{0pt}\setlength{\parskip}{0pt}}
\setcounter{secnumdepth}{0}
% Redefines (sub)paragraphs to behave more like sections
\ifx\paragraph\undefined\else
\let\oldparagraph\paragraph
\renewcommand{\paragraph}[1]{\oldparagraph{#1}\mbox{}}
\fi
\ifx\subparagraph\undefined\else
\let\oldsubparagraph\subparagraph
\renewcommand{\subparagraph}[1]{\oldsubparagraph{#1}\mbox{}}
\fi

%%% Use protect on footnotes to avoid problems with footnotes in titles
\let\rmarkdownfootnote\footnote%
\def\footnote{\protect\rmarkdownfootnote}

%%% Change title format to be more compact
\usepackage{titling}

% Create subtitle command for use in maketitle
\newcommand{\subtitle}[1]{
  \posttitle{
    \begin{center}\large#1\end{center}
    }
}

\setlength{\droptitle}{-2em}

  \title{Программа по эконометрике}
    \pretitle{\vspace{\droptitle}\centering\huge}
  \posttitle{\par}
    \author{Борис Демешев}
    \preauthor{\centering\large\emph}
  \postauthor{\par}
      \predate{\centering\large\emph}
  \postdate{\par}
    \date{2019-04-04}

\newfontfamily{\cyrillicfonttt}{Arial}
\newfontfamily{\cyrillicfont}{Arial}
\newfontfamily{\cyrillicfontsf}{Arial}

\begin{document}
\maketitle

\subsection{Темы}

\hypertarget{-1.-----}{%
\subsubsection{Тема 1. Множественная регрессия без статистических
предпосылок}\label{-1.-----}}

Задача оптимизации. Правила работы с матричным дифференциалом.
Геометрическая интерпретация. Показатели RSS, ESS, TSS, R\^{}2.

\hypertarget{-2.------}{%
\subsubsection{Тема 2. Множественная регрессия с предпосылками на
дисперсию}\label{-2.------}}

Свойства ковариационных матриц. Теорема Гаусса-Маркова с
доказательством.

\hypertarget{-3.-----}{%
\subsubsection{Тема 3. Множественная регрессия и нормальные
остатки}\label{-3.-----}}

Тестирование гипотезы об отдельном коэффициенте. Ограниченная и
неограниченная модель. F-тест для выбора вложенных моделей.

\hypertarget{-4.-}{%
\subsubsection{Тема 4. Гетероскедастичность}\label{-4.-}}

Нахождение эффективных оценок при известной форме гетероскедастичности.
Корректировка стандартных ошибок при неизвестной форме
гетероскедастичности. Тест Уайта на гетероскедастичность

\hypertarget{-5.---}{%
\subsubsection{Тема 5. Метод максимального правдоподобия}\label{-5.---}}

Идея метода максимального правдоподобия. Информация Фишера. Тест
отношения правдоподобия. Тест Вальда

\hypertarget{-6.---}{%
\subsubsection{Тема 6. Модель логистической регрессии}\label{-6.---}}

Предпосылки модели. Оценка коэффициентов и их стандартных ошибок.
Интерпретация коэффициентов. Нахождение предельных эффектов.

\hypertarget{-7.----}{%
\subsubsection{Тема 7. Модели одномерных временных
рядов}\label{-7.----}}

ARIMA-модель. ETS-модель. Уравнение модели. Алгоритм подбора
гиперпараметров. Прогнозирование в рамках модели.

\subsection{Литература}

\begin{itemize}
\item
  Никита Артамонов, Введение в эконометрику. Курс лекций.
\item
  Kurt Schmidheiny, Short guides on econometrics,
  \href{http://schmidheiny.name/teaching/shortguides.htm}{schmidheiny.name/teaching/shortguides.htm}
\item
  Michael Creel, Econometric Lecture notes,
  \href{http://econpapers.repec.org/paper/aubautbar/575.03.htm}{econpapers.repec.org/paper/aubautbar/575.03.htm}
\item
  Материалы курса эконометрики ВШЭ,
  \href{https://bdemeshev.github.io/em301/}{bdemeshev.github.io/em301/}
\end{itemize}

\hypertarget{-}{%
\subsection{Формула оценивания}\label{-}}

Итоговая оценка = 0.4 * Домашнее задание + 0.6 * Письменный экзамен

\hypertarget{--}{%
\subsection{Пример домашнего задания}\label{--}}

Пройдите курсы на datacamp:

\url{https://www.datacamp.com/courses/introduction-to-the-tidyverse},

\url{https://www.datacamp.com/courses/multiple-and-logistic-regression},

\url{https://www.datacamp.com/courses/communicating-with-data-in-the-tidyverse},

\url{https://www.datacamp.com/courses/forecasting-using-r}

\hypertarget{-}{%
\subsection{Пример экзамена}\label{-}}

Задача 1.

Майор пронин наблюдает хочет оценить модель \(y_t = \beta x_t + u_t\),
величины \(u_t\) независимы с нулевым ожиданием и дисперсией
пропорциональной \(1/x_t\). Известны наблюдения:

\begin{longtable}[]{@{}llll@{}}
\toprule
\(y_t\) & 1 & 2 & 3\tabularnewline
\midrule
\endhead
\(x_t\) & 1 & 1 & 2\tabularnewline
\bottomrule
\end{longtable}

\begin{enumerate}
 \item Оцените параметр $\beta$ с помощью МНК. Является ли оценка $\hat\beta_{ols}$ несмещённой? 
 \item Найдите самую эффективную линейную оценку параметра $\beta$, $\hat\beta_{eff}$? 
 \item Найдите стандартную ошибку $se(\hat\beta_{eff})$.
\end{enumerate}

Задача 2.

Билл Гейтс оценил регрессию \(\hat Y_i = 4 + 0.4 X_i + 0.9 W_i\),
\(RSS = 520\), \(R^2 = 2/15\).

Про матрицу регрессоров \(X\) известно, что \[
X'X = \begin{pmatrix}
    29 & 0 & 0 \\
    0 & 50 & 10 \\
    0 & 10 & 80 \\
\end{pmatrix}
\]

\begin{enumerate}
 \item Сколько наблюдений было у Билла Гейтса?
 \item Найдите выборочное среднее переменных $X$, $W$ и $Y$.
 \item Постройте 95\%-й доверительный интервал для фактического значения зависимой переменной при $X=1$ и $W=3$.
\end{enumerate}

Задача 3.

По 200 наблюдениям исследователь Иннокентий строит модель зависимости
финального балла за метрику от числа часов подготовки для разных
студентов.

Он оценил одну и ту же модель на трёх выборках

\begin{longtable}[]{@{}llll@{}}
\toprule
выборка & уравнение & \(RSS\) & наблюдений\tabularnewline
\midrule
\endhead
все студенты & \(\hat y_i = 40 + 2x_i\) & 7000 & 200\tabularnewline
любители пиццы & \(\hat y_i = 50 + 3x_i\) & 3000 & 100\tabularnewline
нелюбители пиццы & \(\hat y_i = 37 + 1.8x_i\) & 2000 &
100\tabularnewline
\bottomrule
\end{longtable}

\begin{enumerate}
    \item Переменная $d_i$ равна 1 для любителей пиццы и 0 для остальных студентов.
Какие оценки коэффициентов получит Иннокентий при оценке модели по всей выборке?
    \[
    \hat y_i = \hat \beta_1 + \hat \beta_2 x_i + \hat \beta_3 d_i + \hat\beta_4 d_i x_i
    \]
    \item Протестируйте гипотезу о том, что симпатия к пицце не влияет на ожидаемый балл по эконометрике.
\end{enumerate}

Задача 4.

Джеймс Бонд наблюдает значения независимых случайных величин \(y_1\),
\(y_2\), \ldots, \(y_{400}\), закон распределения которых задан
табличкой:

\begin{longtable}[]{@{}lccc@{}}
\toprule
\(y_i\) & 1 & 2 & 3\tabularnewline
\midrule
\endhead
вероятность & \(2a\) & \(3a\) & \(1-5a\)\tabularnewline
\bottomrule
\end{longtable}

Из 400 наблюдений: \(100\) наблюдений равны \(1\), \(100\) наблюдений
равны \(2\), и \(200\) наблюдений равны \(3\).

\begin{enumerate}
 \item Оцените неизвестный параметр $a$ методом моментов и методом максимального правдоподобия.
 \item Постройте 95\%-й доверительный интервал для параметра $a$.
\end{enumerate}

Задача 5.

По 200 наблюдениям исследователь Иннокентий оценил модель логистической
регрессии для вероятности сдать экзамен по метрике: \[
\hat P(Y_i = 1) = \Lambda(1.5 + 0.3X_i - 0.4 D_i),
\] где \(Y_i\) --- бинарная переменная равная 1, если студент сдал
экзамен; \(X_i\) --- количество часов подготовки студента; \(D_i\) ---
бинарная переменная равная 1, если студент любит пиццу.

Оценка ковариационной матрицы оценок коэффициентов имеет вид:

\[
\begin{pmatrix}
0.04 & -0.01 & 0 \\
-0.01 & 0.01 & 0 \\
0 & 0 & 0.09 \\
\end{pmatrix}
\]

\begin{enumerate}
  \item Проверьте гипотезу о том, что количество часов подготовки не влияет на вероятность сдать экзамен.
  \item Посчитайте предельный эффект увеличения каждого регрессора на вероятность сдать экзамен для студента не любящего пиццу и готовившегося 24 часа.
Кратко, одной-двумя фразами, прокомментируйте смысл полученных цифр.
 \item Постройте 95\%-й доверительный интервал для разницы вероятностей сдать экзамен двумя студентами, если оба студента готовились 20 часов, однако один любит пиццу, а второй — нет.
  \item При каком значении $D_i$ предельный эффект увеличения $X_i$ на вероятность сдать экзамен максимален,
  если $X_i=20$?
\end{enumerate}


\end{document}
